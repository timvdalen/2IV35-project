\section{Design}\label{Sec:Des}
	\subsection{Visualization Techniques}
	We have three attributes that we want to display, which are:
	\begin{itemize}
	\item \textbf{Year}\\
	The year in which the movie was released. This is a ordered key attribute.
	\item \textbf{Genre} \\
	The genre of the movie, which is a categorical key attribute. 
	\item \textbf{Total} \\
	The number of movies produced in that year for that genre. 
	This is a categorical key attribute.
	\end{itemize}
	So a streamgraph would be ideal for this. 
	However there are still some issues with streamgraphs. 
	As Lee Byron and Martin Wattenberg stated in the conclusion of their paper: 
	\emph{'An important purpose of this paper is to spotlight stacked graphs as an interesting object of study. 
	There are many unresolved questions in their design and evaluation.'}
	This means we have to analyse the streamgraph well. 
	Then we can draw conclusions for which tasks it worked well and for which it did not. 
	TODO reference maken: Stacked Graphs – Geometry and Aesthetics, Lee Byron and Martin Wattenberg

	Also a lot of questions in the tasks are about individual genres. 
	But instead of the total number of movies in that movie we want to look at some other values.
	We want to see the percentage of movies from the total number of movies.
	If we just take the total number of movies, we will see all movies increase in produced movies. 
	This is because nowadays there is a bigger total of movies.
	
	We also want to look at the average rating, because the rating is a interesting way to find out how people perceived the movies. 
	Both the average rating and percentage of the total movies for a genre over time are interesting to visualize. 
	
	This means we have two attributes we want to visualize: 
	\begin{itemize}
	\item \textbf{Year}\\
	The year in which the movie was released. This is a ordered key attribute.
	\item \textbf{Rating/Percentage} \\
	The percentage of movies of that genre from the the total movies produced in that year and the average rating for that movie produced in hat year.
	They are both categorical key attribute.
	\end{itemize}
	So we decided to make two line graphs. 
	One for the rating and one for the percentage of the total number of movies.
	
	We also decided to add a bar chart that shows the distribution of the movies in that year. 
	We think this might be useful while analysing the data to look at the distrubition compared to other genres. 
	
	\subsection{Interaction}
	\textbf{Select}\\
	\emph{Click}\\
	Go to the line charts for that genre and the bar chart for that year. \\
	\emph{Hover}\\
	Show information for that movie for that genre for that year. 
	This hover is shown in Figure~\ref{fig:des1}.
		\begin{figure}[H]
			\centering
			\includegraphics[width=0.8\textwidth]{{des_hover1c}}
			\caption{The hover for the streamgraph}
			\label{fig:des1}
		\end{figure}
	\emph{Highlight}\\
	Change visual encoding for selection targets. 
	You can see the difference between Figure~\ref{fig:des2} and Figure~\ref{fig:des3}.\\
			\begin{figure}[H]
			\centering
			\includegraphics[width=0.4\textwidth]{{des_hover1a}}
			\caption{Highlighted streamgraph}
			\label{fig:des2}
			\end{figure}
			
			\begin{figure}[H]
			\centering
			\includegraphics[width=0.4\textwidth]{{des_hover1b}}
			\caption{Normal streamgraph}
			\label{fig:des3}
			\end{figure}
	\textbf{Change viewpoint }\\
	\emph{Zoom}\\
	Zoom in to inspect some years closer. 
	Use the mouse to move left or right on the graph. \\

	\subsection{Data Representation}
	We decided to preprocess the data. We did this to filter movies out of the list and to save time.
	Since we mainly focus on the genre of movies we removed all movies without genre.
	We also use the release data, so we filtered the movies out that did not have a release date.
	And finally we removed all movies that did not have an IMDb rating. \\
	Since we mainly look at the data at a certain year, we decided to already compute the most used attributes for that year.
	We decided to compute the following attributes for each year for each genre:
	\begin{itemize}
	\item \textbf{Total}\\
	The number of movies produced in that year for that genre.
	\item \textbf{Rating} \\
	The average rating for that movie produced in hat year.
	\item \textbf{Percentage} \\
	The percentage of movies of that genre from the the total movies produced in that year.
	\end{itemize}
	These three attributes take a lot of time to compute, while they are all always the same for that year. 

	