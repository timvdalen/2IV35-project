\section{Conclusions}\label{Sec:Con}
Maximum intensity projection is useful when you are looking for some objects that are hidden.
However the objects that are you looking for must have an high voxel value, otherwise you will not find them.
We were looking for hidden objects in the backpack. 
These objects had high voxel values, so this worked out really well.
However we were also looking for the coins inside the pig.
This did not work out, because the voxel values of the coins were too low. 
Another problem with maximum intensity projection is that is does not display depth. \\
The composite transfer function is useful when you want too see through semi-transparant layers. 
The main advantages of the composite transfer function is that there is not much loss of information.
You use all layers to generate the new image. 
However the biggest problem of the composite transfer function is that you have to experiment with the settings of the transfer function.
With the proper settings for the transfer function the composite transfer function really excels. \\
The first technique did worked out pretty well. It is useful to easily find out how the surface looks like.
It is also useful when you want to find object with an high voxel value. 
However maximum intensity projection is better at finding an object with an high voxel value.
And when you want to know how the surface looks like, there is no need to turn it in a volume file. \\
Gradient based opacity weighting is useful when there are different kinds of materials with voxel valuse in other ranges.
Then you can use gradient based opacity weighting to show only these materials.
This turned out useful with the carp and the stent. 
However it did not really work well with the coins in the pig, because they were in the same voxel value range.



