\section{City-Data}
This appendix shows an detailed description of  \emph{ cities data.txt} as given in the assignment. 
\textbf{GM CODE} $[string]$ \\
This code provides a numerical identity to a municipality. \\
\textbf{GM NAAM} $[string]$ \\
The official name of the municipality. \\
\textbf{OAD} $[number]$ \\
Average number of addresses per square kilometer. \\
\textbf{STED} $[number]$ \\
Describes the urban character using the following classification: \\
1 very strongly urban  more than 2500 addresses per km2 \\
2 strongly urban between 1500 and  2500 addresses per km2 \\
3 moderately urban between 1000 and 1500 addresses per km2 \\
4 slightly urban between 500 and  1000 addresses per km2 \\
5 not urban < 500 addresses per km2 \\
\textbf{AANT INW}$ [number] $ \\
Number of inhabitants. \\
\textbf{AANT MAN}$ [number]$ \\
Number of men. \\
\textbf{AANT VROUW}$ [number]$ \\
Number of women. \\
\textbf{P 00 14 JR}$ [percentage]$ \\
Percentage of inhibations aged 0 to 15 years. \\
\textbf{P 15 24 JR}$ [percentage]$ \\
Percentage of inhibations aged 15 to 25 years. \\
\textbf{P 25 44 JR}$ [percentage]$ \\
Percentage of inhibations aged 25 to 45 years. \\
\textbf{P 45 64 JR}$ [percentage]$ \\
Percentage of inhibations aged 45 to 65 years. \\
\textbf{P 65 EO JR}$ [percentage]$ \\
Percentage of inhibations aged 65 years and older. \\
\textbf{P ONGEHUWD}$ [percentage]$ \\
Percentage of unmarried people. \\
\textbf{P GEHUWD}$ [percentage]$ \\
Percentage of married people. \\
\textbf{P GESCHEID}$ [percentage]$ \\
Percentage of divorced people. \\
\textbf{P VERWEDUW}$ [percentage]$ \\
Percentage of widows and widowers. \\
\textbf{BEV DICHTH}$ [number]$ \\
Number of inhabitants per km2. \\
\textbf{AANTAL HH}$ [number]$ \\
Number of households. \\
\textbf{P EENP HH}$ [percentage]$ \\
Percentage of single households. \\
\textbf{P HH Z K}$ [percentage]$ \\
Percentage of households without children. \\
\textbf{P HH M K}$ [percentage]$ \\
Percentage of households with children. \\
\textbf{GEM HH GR}$ [number]$ \\
Average number of people in all households. \\
\textbf{P WEST AL}$ [percentage]$ \\
Percentage of foreigners from Europe, North-America, Oceania, Indonesia, and Japan. \\
\textbf{P N W AL}$ [percentage]$ \\
Percentage of foreigners not from Europe, North-America, Oceania, Indonesia, and Japan. \\
\textbf{P MAROKKO}$ [percentage]$ \\
Percentage of foreigners from Morocco, Spanish Sahara, and Western Sahara. \\
\textbf{P ANT ARU}$ [percentage]$ \\
Percentage of foreigners from the Dutch Antilles and Aruba. \\
\textbf{P SURINAM}$ [percentage]$ \\
Percentage of foreigners from Surinam. \\
\textbf{P TURKIJE}$ [percentage] $\\
Percentage of foreigners from Turkey. \\
\textbf{P OVER NW}$ [percentage]$ \\
Percentage of foreigners from other countries than mentioned in the above 4 attributes. \\
\textbf{AUTO TOT}$ [number]$ \\
Number of cars. \\
\textbf{AUTO HH}$ [number]$ \\
Number of cars per household. \\
\textbf{AUTO LAND}$ [number]$ \\
Number of cars per km2. \\ 
\textbf{BEDR AUTO}$ [number]$ \\
Number of company cars (minivans, trucks, etc.) \\
\textbf{MOTOR 2W}$ [number]$ \\
Number of motorcycles, including scooters. \\
\textbf{OPP TOT}$ [number]$ \\
Total land and water area in hectares. \\
\textbf{OPP LAND}$ [number]$ \\
Land area in hectares. \\
\textbf{OPP WATER}$ [number] $\\
Water area in hectares. \\
\textbf{P 00 04 JR}$ [percentage]$ \\
Percentage of inhibations aged 0 to 5 years. \\
\textbf{P 05 09 JR}$ [percentage]$ \\
Percentage of inhibations aged 5 to 10 years. \\
\textbf{P 10 14 JR}$ [percentage]$ \\
Percentage of inhibations aged 10 to 15 years. \\
\textbf{P 15 19 JR}$ [percentage]$ \\
Percentage of inhibations aged 15 to 20 years. \\
\textbf{P 20 24 JR}$ [percentage]$ \\
Percentage of inhibations aged 20 to 25 years. \\
\textbf{P 25 29 JR}$ [percentage]$\\
Percentage of inhibations aged 25 to 30 years. \\
\textbf{P 30 34 JR}$ [percentage]$ \\
Percentage of inhibations aged 30 to 35 years. \\
\textbf{P 35 39 JR}$ [percentage]$ \\
Percentage of inhibations aged 35 to 40 years. \\
\textbf{P 40 44 JR}$ [percentage]$ \\
Percentage of inhibations aged 40 to 45 years. \\
\textbf{P 45 49 JR}$ [percentage]$ \\
Percentage of inhibations aged 45 to 50 years. \\
\textbf{P 50 54 JR}$ [percentage]$ \\
Percentage of inhibations aged 50 to 55 years. \\
\textbf{P 55 59 JR}$ [percentage]$\\
Percentage of inhibations aged 55 to 60 years. \\
\textbf{P 60 65 JR}$ [percentage]$ \\
Percentage of inhibations aged 60 to 65 years.\\
\textbf{P 65 69 JR}$ [percentage]$ \\
Percentage of inhibations aged 65 to 70 years. \\
\textbf{P 70 74 JR}$ [percentage]$ \\
Percentage of inhibations aged 70 to 75 years. \\
\textbf{P 75 79 JR}$ [percentage]$ \\
Percentage of inhibations aged 75 to 80 years. \\
\textbf{P 80 84 JR}$ [percentage]$ \\
Percentage of inhibations aged 80 to 85 years. \\
\textbf{P 85 89 JR}$ [percentage]$ \\
Percentage of inhibations aged 85 to 90 years. \\
\textbf{P 90 94 JR}$ [percentage]$ \\
Percentage of inhibations aged 90 to 95 years. \\
\textbf{P 95 EO JR}$ [percentage]$ \\
Percentage of inhibations aged 95 years and older. \\