\section{Design}\label{Sec:Des}
\subsection{General Design}
We want two views, which have some interaction between each other. The first view, that was given as an example with the assignment displays all the municipalities on a map of the Netherlands. We will call this view the main view. The reason we still want the use the map is because we think that users who are interested in the use of the application know the location of the municipality they are interested in. This means he can quickly find the location and read it's value and compare it too other municipalities. Of course not everyone knows which municipalities is displayed on the map exactly, thus hovering over a municipality will show the name and exact value of that municipality. It is not really fast to hover over all municipalities if you do not know where it lies on the map at all, thus we are planning on adding a filter, which enables users to find municipalities really fast. \\
We expect the differences to be small between the provinces, so it could come in handy to know if the province is above or below average. This can be hard to see with just the colour, because all colours look a lot like each other. Especially when you have two municipalities of the same colour, but one is surrounded by darker colours and the other is surrounded by lighter colours the user might get fooled by the contrast illusion. That's why we are planning to add a hover to the legend. When you hover over a province or municipality then there is also a hover over the legend, that shows the average age of that province or municipality.\\
In the other view we want to display a bar chart with data of that municipality.  We want to do this, because when reading different percentages it is hard to compare different percentages. With a bar chart that displays the percentage of each category you can easily which group is bigger than the others or which one is the smallest. We will call this view the detail view. Of course we want some interaction between the detail view and the main view. We want to show and or update the detail view each time the user clicks on a municipality in the main view. \\
Another visualization technique we want to apply is aggregation. The reason we want to apply this technique is, because there are over 400 different municipalities. This makes it really hard to compare these municipalities and read the results of the map. So we wanted to bring down the number of different locations down. We thought that logical combinations of municipalities would be the twelve provinces from the Netherlands. Then it would be a lot easier to compare the data of each province to the others and still have a pretty good idea how the data is diffused in the Netherlands. \\
We know this could lead to a loss of data, which is not what exactly what we want. The user of the application should still be able to find the data of each municipality. To show the data of each municipality, we choose to make an interaction with the map. Clicking on a province will ''expand'' that province and instead of showing the data of that province, show the data of all municipalities in that province. This way we make sure the user can compare the different areas of the Netherlands quickly, without any loss of data. \\
\subsection{Data Representation}
To display whole provinces, we had to combine the area of each that municipality in that province. First we had to convert the data of the CBS to data that is in the same format. When we add GM to all ID's in the they got in the same format. So we now have a list of id's for each municipality for each province. We then wrote a simple javascript file that would iterate over each province and add the geometry of each municipality to the correct provinces. \\
Unfortunately we ran in a problem here. The data with geolocation of each municipalities contained municipalities that no longer existed. This municipalties did not existed in the province-municipalities list, but they also were missing in the cities-data. For making the geometry for the province this was not a real big problem and we looked up the municipalities that were missing and added them manually to the correct province. \\
Now we had the geometry of each problem, we still had to remove the borders of each municipality and add a new border around the whole province. To do this we used \emph{Turf-merge}.  \emph{Turf-merge} is a plugin for NPM, which merges multiple geolocations to one geolocation. It worked pretty well, but there were still some borders left in the middle of the province. There were also two municipalities that would not merge with \emph{Turf-merge}. We removed this municipalties and draw them with a square as they were in the middle of the province and merge that square instead of the municipality which it would merge. \\
After we got the three correct datasets, we thought that it would be nice to combine those three, so we it is easier to work with the data. When we have all data in one file, we can easily look up the values, geometry and province. This is a lot faster than looping through three different files to find each municipality three times. So we made one json file that contained all the data. The format of this file is simple: \\
We have an array of Provinces. Each provinces has multiple attributes namely a name, a geometry and a list of municipalities. Each municipality has a code, a name, a geometry and the all the attributes from \emph{ cities data.txt}.
